\documentclass{article}
\usepackage{graphicx} 				%simple figure inclusion
\usepackage{epstopdf, epsfig} 		%simple figure inclusion
\usepackage{amsmath, amssymb, mathtools} %math
\usepackage[utf8]{inputenc} 		%idk
\usepackage[english]{babel} 		%idk
\usepackage{microtype} 				%idk
\usepackage[left=1.5in, right=1.5in]{geometry}
\usepackage{fancyhdr} 				%header footer
%\usepackage{mathptmx} 				%Times New Roman
\usepackage[nottoc]{tocbibind} 		%place ToC and References
\usepackage{float}					%place floats precisely
\usepackage{tabulary}				%tables with good column size
\usepackage[numbers]{natbib}		%bibliography with numbers, alpha order with APA
\usepackage{hyperref}				%clickable references
\usepackage{xfrac}   				%for \sfrac macro
\usepackage[final]{pdfpages}		%add pdf pages
\usepackage{bm}						%"more reliable" math boldface
\usepackage{soul}					%smart underlining
\usepackage{listings}				%include code

\pagestyle{fancy}
\cfoot{\thepage}

\hypersetup{hidelinks}			%no box around links

\linespread{1.3}

\newcommand*{\Perm}[2]{{}^{#1}\!P_{#2}}%


%New colors defined below
\definecolor{codegreen}{rgb}{0,0.6,0}
\definecolor{codegray}{rgb}{0.5,0.5,0.5}
\definecolor{codepurple}{rgb}{0.58,0,0.82}
\definecolor{backcolour}{rgb}{0.95,0.95,0.92}

%Code listing style named "mystyle"
\lstdefinestyle{mystyle}{
  backgroundcolor=\color{backcolour},   commentstyle=\color{codegreen},
  keywordstyle=\color{magenta},
  numberstyle=\tiny\color{codegray},
  stringstyle=\color{codepurple},
  basicstyle=\footnotesize,
  breakatwhitespace=false,         
  breaklines=true,                 
  captionpos=b,                    
  keepspaces=true,                 
  numbers=left,                    
  numbersep=5pt,                  
  showspaces=false,                
  showstringspaces=false,
  showtabs=false,                  
  tabsize=2
}

%"mystyle" code listing set
\lstset{style=mystyle}
\lstset{basicstyle=\footnotesize\ttfamily,breaklines=true}
\lstset{framextopmargin=50pt,frame=bottomline}

\renewcommand\thesubsection{\Alph{subsection}}

\DeclareMathOperator{\pr}{pr}

%========================================================================

\begin{document}

\title{Summary of Intensity Measures for Collapse and Impact Prediction}
\author{Huy G. Pham}

\maketitle

\section{Intensity Measures}

To properly characterize the intensity measures along with the different isolated buildings, we are interested in not only intensity measures, but non-dimensionalized against building characteristics. We consider two sets of variables: the first to characterize isolated building parameters, and the second to characterize ground motion intensity.

\subsection{Building parameters}
To characterize the building parameters, the considered variables are:

	\begin{enumerate}
		\item Fixed-base structure period, $T_{fb}$. This represents the period of the structure above the isolation layer were it to be built conventionally without isolation. This allows a characterization of the superstructure and potentially impacts superstory drift.
		\item Effective period of the isolation system, $T_M$. This represents the period of the entire isolation system. 
	\end{enumerate}
	
To include these variables in a manner that is appropriately dimensionless, we propose the following measure. Starting from the ASCE 7-16 Eq. 17.5-1 (maximum displacement capacity for an isolated system),

	$$D_M = \frac{g S_{M1} T_M}{4 \pi^2 B_M}$$

We seek to use for $D_M$ our amplified moat gap, which is the exact equation above, amplified with a random scalar from 1.0 to 2.25. We can then back calculate an amplified $S_{M}$ that the structure was designed for. This follows from the simple conversion that $S_M = \frac{S_{1M}}{T_M}$.

	$$S_{M, amp} = \frac{A_\delta D_M 4 \pi^2 B_M}{g T_M^2}$$
	
where $A_\delta D_M$ is the amplified moat gap in the design. We also seek to find the unamplified spectral pseudoacceleration at which the structure is designed for,

	$$S_M = Sa(T_M)$$
	
which is the spectral acceleration at the effective period of the bearing, calculated from the $S_1$ value of the design spectrum that the designer chose.

\subsection{Ground motion IMs}
To characterize the ground motion, the considered IMs are:

	\begin{enumerate}
		\item Peak Ground Acceleration 
			$$PGA = \max_t \left | \ddot{u}_g(t) \right |$$
		\item Peak Ground Velocity 
			$$PGV = \max_t \left | \dot{u}_g(t) \right |$$
		\item Average spectral pseudoacceleration as defined in Eads (2015)
			$$Sa_{avg}(T) = \left ( \prod_{i=1}^N Sa (c_i \cdot T) \right ) ^ {1/N}$$
		where $N$ is the number of ordinates considered (spacing). A choice is $[0.2 T, 3T]$; however, for isolated structures, it is more desirable to use $[0.2 T, 1.5 T]$, since the periods for isolated structures are already high. The choice of $T$ is then selected as a typical average period for isolated structure $T_{isol}$. This was previously chosen as 3.0 seconds, but can be lower based on the range of isolated structures considered.
		\item Spectral pseudoacceleration of a building damped linear elastic SDOF with a period equal to the fixed base period of the structure
			$$Sa(T_{fb}, \zeta = \zeta_M) = \omega_{fb}^2 \cdot \max_t \left | u(t) \right | $$
		This is dependent on knowing the fixed base period of the structure.
		\item Spectral pseudoacceleration of a building damped linear elastic SDOF with a period equal to the post-sliding period of the structure
			$$Sa(T_2, \zeta = \zeta_M) = \omega_2^2 \cdot \max_t \left | u(t) \right | $$
		This is dependent on knowing the post-sliding period of the structure.
		\item Spectral pseudoacceleration of a building damped linear elastic SDOF with a period equal to the effective period of the structure
			$$Sa(T_M, \zeta = \zeta_M) = \omega_M^2 \cdot \max_t \left | u(t) \right | $$
		This is dependent on knowing the period of the structure. This is selected because isolated structures may be influenced by pulses with higher periods (lower frequencies).
		\item Maximum instantaneous power of the ground motion, with respect to the effective period of the structure
			$$ IP(T_M) = \max \left ( \frac{1}{\Delta t} \int_{t}^{t + \Delta t} V_{filtered}^2 (t) dt \right ) $$
		where $\Delta t$, the window for integration is a function of $T_M$, $\Delta t = 0.5 T_M$.
		\item Filtered incremental velocity, with respect to the effective period of the structure
			$$ FIV3(T_M) = \max \left ( V_{s, max, 1} + V_{s, max, 2} + V_{s, max, 3}, \left | V_{s, min, 1} + V_{s, min, 2} + V_{s, min, 3} \right | \right )$$
		where $V_{s, i}$ are the local maxima and minima of $V_s(t)$, and 
			$$ V_s(t) = \int_{t}^{t + 0.7 T_M} \ddot{u}_{gf} (\tau) d \tau $$
		is the integrated velocity from the filtered acceleration of the ground motion.
	\end{enumerate}
	
\subsection{Dimensionless variables}

With both ground motion variables and structure variables set up, we can create dimensionless variables as combinations of the two:

	\begin{enumerate}
		\item $\frac{ Sa_{avg}(T, \zeta = \zeta_M) }{ S_{M, amp} }$
		\item $\frac{ Sa_{avg}(T, \zeta = \zeta_M) }{ S_{M} }$
		\item $\frac{ Sa(T_2, \zeta = \zeta_M) }{ S_{M, amp} }$
		\item $\frac{ Sa(T_2, \zeta = \zeta_M) }{ S_{M} }$
		\item $\frac{ Sa(T_M, \zeta = \zeta_M) }{ S_{M, amp} }$
		\item $\frac{ Sa(T_M, \zeta = \zeta_M) }{ S_{M} }$
		\item $\frac{IP(T_M)}{ S_{M, amp} \cdot  D_M}$
		\item $\frac{PGA}{ S_{M, amp} \cdot g}$
		\item $\frac{PGV}{ S_{M, amp} \cdot T_M \cdot g}$
		\item $\frac{FIV3}{ S_{M, amp} \cdot T_M \cdot g}$
		\item $\frac{IP(T_M)}{ S_{M} \cdot  D_M}$
		\item $\frac{FIV3}{ S_{M} \cdot T_M \cdot g}$
	\end{enumerate}
	


\end{document}